%%%%%%%%%%%%%%%%%%%%%%%%%%%%%%%%%%%%%%%
% Deedy - One Page Two Column Resume
% LaTeX Template
% Version 1.2 (16/9/2014)
%
% Original author:
% Debarghya Das (http://debarghyadas.com)
%
% Original repository:
% https://github.com/deedydas/Deedy-Resume
%
% IMPORTANT: THIS TEMPLATE NEEDS TO BE COMPILED WITH XeLaTeX
%
% This template uses several fonts not included with Windows/Linux by
% default. If you get compilation errors saying a font is missing, find the line
% on which the font is used and either change it to a font included with your
% operating system or comment the line out to use the default font.
% 
%%%%%%%%%%%%%%%%%%%%%%%%%%%%%%%%%%%%%%
% 
% TODO:
% 1. Integrate biber/bibtex for article citation under publications.
% 2. Figure out a smoother way for the document to flow onto the next page.
% 3. Add styling information for a "Projects/Hacks" section.
% 4. Add location/address information
% 5. Merge OpenFont and MacFonts as a single sty with options.
% 
%%%%%%%%%%%%%%%%%%%%%%%%%%%%%%%%%%%%%%
%
% CHANGELOG:
% v1.1:
% 1. Fixed several compilation bugs with \renewcommand
% 2. Got Open-source fonts (Windows/Linux support)
% 3. Added Last Updated
% 4. Move Title styling into .sty
% 5. Commented .sty file.
%
%%%%%%%%%%%%%%%%%%%%%%%%%%%%%%%%%%%%%%%
%
% Known Issues:
% 1. Overflows onto second page if any column's contents are more than the
% vertical limit
% 2. Hacky space on the first bullet point on the second column.
%
%%%%%%%%%%%%%%%%%%%%%%%%%%%%%%%%%%%%%%


\documentclass[]{deedy-resume-openfont}
\usepackage{fancyhdr}
    
\pagestyle{fancy}
\fancyhf{}
    
\begin{document}

%%%%%%%%%%%%%%%%%%%%%%%%%%%%%%%%%%%%%%
%
%     LAST UPDATED DATE
%
%%%%%%%%%%%%%%%%%%%%%%%%%%%%%%%%%%%%%%
\lastupdated

%%%%%%%%%%%%%%%%%%%%%%%%%%%%%%%%%%%%%%
%
%     TITLE NAME
%
%%%%%%%%%%%%%%%%%%%%%%%%%%%%%%%%%%%%%%
\namesection{黄}{峻涛}{ \urlstyle{same}\href{mailto:creeperhjt@sjtu.edu.cn}{creeperhjt@sjtu.edu.cn} | 1895 0071 508
}

%%%%%%%%%%%%%%%%%%%%%%%%%%%%%%%%%%%%%%
%
%     COLUMN ONE
%
%%%%%%%%%%%%%%%%%%%%%%%%%%%%%%%%%%%%%%

\begin{minipage}[t]{0.25\textwidth} 

%%%%%%%%%%%%%%%%%%%%%%%%%%%%%%%%%%%%%%
%     EDUCATION
%%%%%%%%%%%%%%%%%%%%%%%%%%%%%%%%%%%%%%

\section{教育经历} 
\sectionsep

\subsection{上海交通大学}
\sectionsep
\descript{在读本科生,软件工程}
\location{2022.09 - 至今}
\location{预计 2026.06毕业}
\sectionsep
\descript{GPA: 3.90/4.30(前 20\%)}
\sectionsep

%%%%%%%%%%%%%%%%%%%%%%%%%%%%%%%%%%%%%%
%     LINKS
%%%%%%%%%%%%%%%%%%%%%%%%%%%%%%%%%%%%%%

\section{链接}
\sectionsep
Blog://  \href{https://creeper12356.github.io/}{\bf creeper12356} \\ 
Github:// \href{https://github.com/creeper12356}{\bf creeper12356} \\

%%%%%%%%%%%%%%%%%%%%%%%%%%%%%%%%%%%%%%
%     COURSEWORK
%%%%%%%%%%%%%%%%%%%%%%%%%%%%%%%%%%%%%%

% \section{修读课程}
% \subsection{Graduate}
% Advanced Machine Learning \\
% Open Source Software Engineering \\
% Advanced Interactive Graphics \\
% Compilers + Practicum \\
% Cloud Computing \\
% Evolutionary Computation \\
% Defending Computer Networks \\
% Machine Learning \\
% \sectionsep

%%%%%%%%%%%%%%%%%%%%%%%%%%%%%%%%%%%%%%
%     SKILLS
%%%%%%%%%%%%%%%%%%%%%%%%%%%%%%%%%%%%%%

\section{技能}
\sectionsep
\subsection{编程}
\location{超过 5000 行}
C++ \textbullet{} Java \\
\location{1000 - 5000 行}
C \textbullet{} Python \\
\location{低于 1000 行}
HTML \textbullet{} Javascript \textbullet{} MatLab \\ 
\sectionsep

\subsection{工具}
Git \textbullet{} Vim \textbullet{} Shell \textbullet{} GDB \\
\sectionsep

\subsection{架构}
\location{了解}
Microservice \textbullet{} Serverless \textbullet{} SQL DB
\textbullet{} Docker \textbullet{} RESTful \textbullet{}
GraphQL \textbullet WebSocket \\
\location{一般}
Kafka \textbullet{} Redis \textbullet{} Nginx \textbullet NoSQL DB \textbullet{} Neo4j \textbullet HDFS
\sectionsep

%%%%%%%%%%%%%%%%%%%%%%%%%%%%%%%%%%%%%%
%
%     COLUMN TWO
%
%%%%%%%%%%%%%%%%%%%%%%%%%%%%%%%%%%%%%%

\end{minipage} 
\hfill
\begin{minipage}[t]{0.73\textwidth} 

%%%%%%%%%%%%%%%%%%%%%%%%%%%%%%%%%%%%%%
%     EXPERIENCE
%%%%%%%%%%%%%%%%%%%%%%%%%%%%%%%%%%%%%%

% \section{实习经历}
% \sectionsep
% \runsubsection{谷歌编程之夏}
% \descript{学生参与者}
% \location{2017.05 - 2017.09 | 远程}
% \vspace{\topsep}
% \begin{tightemize}
%     \item 共有 20651 个注册学生,其中 1318 个申请被谷歌接收,\textbf{接收率 6\%}
%     \item 为 Processing 基金会实现了对 Processing 的 R 语言支持
%     \item 与社区紧密合作,实现对 Processing 库的支持和对 R 包的支持
%     \item 所做项目 \href{https://github.com/gaocegege/Processing.R}{\bf Processing.R} 在 GitHub 上获得 \textbf{70 stars},成为本次编程之夏 star 最多的项目
% \end{tightemize}
% \sectionsep

% \runsubsection{摩根士丹利}
% \descript{CIP 项目实习生}
% \location{2017.02-2017.08 | 上海}
% \begin{tightemize}
% \item 优化开源容器调度管理框架 treadmill 的调度器
% \item 实现与 Kubernetes 类似的调度模型,同时保留自身的树形结构
% \end{tightemize}
% \sectionsep

% \runsubsection{蚂蚁金服(杭州)网络技术有限公司}
% \descript{Java 研发工程师(实习)}
% \location{2015.07-2015.09 | 杭州}
% \begin{tightemize}
% \item 在支付宝国际事业团队从事海外直购业务开发
% \item 实现部分包裹清关的逻辑和后台管理的逻辑
% \end{tightemize}
% \sectionsep

%%%%%%%%%%%%%%%%%%%%%%%%%%%%%%%%%%%%%%
%     RESEARCH
%%%%%%%%%%%%%%%%%%%%%%%%%%%%%%%%%%%%%%

\section{项目经历}
\sectionsep
\runsubsection{\href{https://github.com/UniGPT-SJTU/UniGPT-backend-microservice}{\bf UniGPT}}
\descript{Collaborator}
\location{2024.04-2024.08}
\sectionsep
\begin{tightemize}
    \item 一个AI提示词社区后端
    \item 调用LLM API,结合了提示工程,工具调用,无状态云函数、RAG等多种技术
    \item 采用微服务框架,并使用Docker部署上云
    \item 使用Kubernetes构建弹性可伸缩服务
    \item 使用Nginx实现负载均衡与请求转发
    \end{tightemize}
\sectionsep

\runsubsection{\href{https://github.com/creeper12356/online-bookstore-server}{\bf Online Bookstore}}
\descript{Owner}
\location{2024.02-2024.12}
\begin{tightemize}
    \item 一个线上书店后端
    \item 使用Spring Boot框架,实现了用户管理、图书管理、订单管理等功能
    \item 使用MongoDB管理图书评论
    \item 使用Neo4j管理图书关系,实现图书推荐
    \item 使用Docker部署上云
    \item 使用Kafka消息队列实现异步处理、高吞吐量
    \item 使用Redis缓存提高性能
\end{tightemize}
\sectionsep

\runsubsection{\href{https://github.com/creeper12356/SE3331-CSE-chfs}{\bf CHFS}}
\descript{Owner}
\location{2024.10-2024.12}
\begin{tightemize}
    \item 一个分布式inode文件系统
    \item 使用C++开发,实现文件系统基本功能
    \item 效仿GFS,实现master, chunkserver, client的分布式架构
    \item 使用RPC实现master和chunkserver之间的通信
\end{tightemize}
\sectionsep

\runsubsection{\href{https://github.com/creeper12356/LSM-KV}{\bf LSM-KV}}
\descript{Owner}
\location{2024.04-2024.06}
\begin{tightemize}
    \item 基于LSM-Tree的键值分离存储系统
    \item 使用C++开发,实现了get, put, delete, scan等接口
    \item 具有分层架构、键值分离等特点,对写操作有较好性能
\end{tightemize}
\sectionsep


\runsubsection{\href{https://github.com/creeper12356/SE3355-Tiger-Compiler}{\bf Tiger Compiler}}
\descript{Owner}
\location{2024.10-2025.01}
\begin{tightemize}
    \item 基于LLVM后端的Tiger语言编译器
    \item 支持从Tiger语言编译到x86-64汇编,编译后的程序可以在Linux上运行
    \item 实现基本语言特性,如变量、函数、数组、结构体、循环、条件等
    \item 模块化设计,包括词法、语法、语义分析、IR生成、汇编代码生成、寄存器分配
\end{tightemize}
\sectionsep

\runsubsection{\href{https://github.com/creeper12356/AltCampusLife}{\bf AltCampusLife}}
\descript{Owner}
\location{2024.10,不定时维护}
\begin{tightemize}
    \item 电动车充电手机APP
    \item 使用React Native开发,支持Android和iOS
    \item 通过反编译已有APP以及接口测试,逆向分析后端API并对接
    \item 校内有一定用户量
\end{tightemize}
\sectionsep


%%%%%%%%%%%%%%%%%%%%%%%%%%%%%%%%%%%%%%
%     OPEN SOURCE
%%%%%%%%%%%%%%%%%%%%%%%%%%%%%%%%%%%%%%

% \section{开源贡献}
% \begin{tabular}{ll}
% \href{https://github.com/moby/moby/commits?author=gaocegege}{\bf moby/moby} & 实现 docker service ps -q 参数,与 swarmkit 更好集成 \\
% \href{https://github.com/opencontainers/runc/commits?author=gaocegege}{\bf opencontainers/runc} & 为了修复 \href{https://github.com/moby/moby/issues/27484}{moby/moby\#27484} 对上游进行的修改 \\
% \href{https://github.com/pingcap/tidb/commits?author=gaocegege}{\bf pingcap/tidb} & 在 travis 里引入了覆盖率测试; 实现 truncate 函数 \\
% \href{https://github.com/coala/coala-vs-code/commits/master?author=gaocegege}{\bf coala/coala-vs-code} & Visual Studio Code 上的插件,项目 maintainer \\
% \href{https://github.com/weijianwen/SJTUThesis/commits?author=gaocegege}{\bf weijianwen/SJTUThesis} & 为学士论文模板添加英文大摘要; 替换版权字体 \\
% \end{tabular}
% \sectionsep

%%%%%%%%%%%%%%%%%%%%%%%%%%%%%%%%%%%%%%
%     AWARDS
%%%%%%%%%%%%%%%%%%%%%%%%%%%%%%%%%%%%%%

\section{所获奖项} 
\begin{tabular}{rll}
2023         & 上海交通大学C等奖学金 \\
\end{tabular}
\sectionsep

%%%%%%%%%%%%%%%%%%%%%%%%%%%%%%%%%%%%%%
%     PUBLICATIONS
%%%%%%%%%%%%%%%%%%%%%%%%%%%%%%%%%%%%%%

% \section{Publications} 
% \renewcommand\refname{\vskip -1.5cm} % Couldn't get this working from the .cls file
% \bibliographystyle{abbrv}
% \bibliography{publications}
% \nocite{*}

\end{minipage} 
\end{document}  \documentclass[]{article}
